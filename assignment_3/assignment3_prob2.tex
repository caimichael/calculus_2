\documentclass[11pt, oneside]{article}   	% use "amsart" instead of "article" for AMSLaTeX format
\usepackage{geometry}                		% See geometry.pdf to learn the layout options. There are lots.
\geometry{letterpaper}                   		% ... or a4paper or a5paper or ... 
%\geometry{landscape}                		% Activate for rotated page geometry
%\usepackage[parfill]{parskip}    		% Activate to begin paragraphs with an empty line rather than an indent
\usepackage{graphicx}				% Use pdf, png, jpg, or eps§ with pdflatex; use eps in DVI mode
								% TeX will automatically convert eps --> pdf in pdflatex		
\usepackage{amssymb}

%SetFonts

%SetFonts


\title{Assignment 3 Problem 2}
\author{Michael Cai}
\date{\today}							% Activate to display a given date or no date

\begin{document}
\maketitle
\noindent A rumor (that Calculus is a fun class) is spread in a school. The time $t$ at which a fraction $p$ of the school population has heard the rumor is given by: $t(p) = \int_a^p \frac{b}{x(1-x)}dx$, where $a$ and $b$ are constants, with $0<a<1$ and $b>0$. \\~\\
(a) Evaluate the integral to find an explicit formula for $t(p)$. Write your answer so that it only has one ln term.\\
Because the integral is in the form of a rational function, we use Partial Fraction Decomposition.\\
$\frac{b}{x(1-x)} = \frac{A}{x} + \frac{B}{1-x}$\\
Cross-multiplying and comparing terms to the numerator on the LHS we get:\\
$b = A(1-x) + B(x)$\\
When $x=1$, $B=b$\\
When $x=0$, $A=b$\\
Therefore, $\frac{b}{x(1-x)} = \frac{b}{x} + \frac{b}{1-x}$\\
And thus, $\int_a^p \frac{b}{x(1-x)} = \int_a^p \frac{b}{x} + \int_a^p \frac{b}{1-x}$\\
$= \left. blnx \right|_a^p - \left. bln(1-x) \right|_a^p$\\
$= blnp - bln(1-p) + bln(1-a) - blna$\\
$= bln(\frac{p}{1-p}) + bln(\frac{1-a}{a})$\\
$= bln(\frac{p(1-a)}{a(1-p)})$\\~\\
(b) It turns out that at time $t=0$, one percent of the school population ($p=0.01$) has heard the rumor. Solve for the constant $a$. \\
$t(p) = t(0.01) = 0 = bln(\frac{p(1-a)}{a(1-p)})$\\
$ln(\frac{0.01 - 0.01a}{a - 0.01a}) = 0$\\
$ln(0.01 - 0.01a)-ln(a-0.01a) = 0$\\
$ln(0.01 - 0.01a) = ln(a-0.01a)$\\
$e^{ln(0.01 - 0.01a)} = e^{ln(a-0.01a)}$\\
$0.01 - 0.01 a = a - 0.01a$\\
$a = 0.01$\\
This makes sense because for an integral to equal 0 the lower and upper bounds of the integral must equal the same number. Thus $a$ must equal $p$ for $t$ to equal 0.\\~\\
(c) In addition to the information in part (b) above, it turns out that at time $t=1$, half the school population has heard the rumor. Solve for the constant $b$. \\
$1 = bln(\frac{0.5(1-0.01)}{0.01(1-0.5)})$\\
$\frac{1}{b} = ln(\frac{.495}{.005})$\\
$\frac{1}{b} = ln(99)$\\
$b = \frac{1}{ln99}$\\~\\
(d) At what time has 90 percent of the school population heard the rumor?\\
$t = \frac{1}{ln99}(ln(\frac{.9(1-.01)}{.01(1-.9)}))$\\
$t = \frac{1}{ln99}(ln(\frac{.891}{.001}))$\\
$t = \frac{ln891}{ln99}$\\
$t \approx 1.4782$\\
This means that at time $t=1.4782$, 90 percent of the school population has heard the rumor.\\

\end{document}  