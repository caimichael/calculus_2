\documentclass[11pt, oneside]{article}   	% use "amsart" instead of "article" for AMSLaTeX format
\usepackage{geometry}                		% See geometry.pdf to learn the layout options. There are lots.
\geometry{letterpaper}                   		% ... or a4paper or a5paper or ... 
%\geometry{landscape}                		% Activate for rotated page geometry
%\usepackage[parfill]{parskip}    		% Activate to begin paragraphs with an empty line rather than an indent
\usepackage{graphicx}				% Use pdf, png, jpg, or eps§ with pdflatex; use eps in DVI mode
								% TeX will automatically convert eps --> pdf in pdflatex		
\usepackage{amssymb}

%SetFonts

%SetFonts


\title{Assignment 3 Problem 4}
\author{Michael Cai}
\date{\today}							% Activate to display a given date or no date

\begin{document}
\maketitle

\noindent Consider the following integral:\\
$\int_0^{\pi/2} \frac{sin^n(x)}{sin^n(x) + cos^n(x)} dx$ where $n$ is any positive integer.\\

\noindent (a) \textit{Use Wolfram Alpha or another computer algebra system to evaluate the above integral (for a general n). Write down the result.}\\
Standard computational time exceeded.\\

\noindent (b) \textit{Using a computer algebra system, evaluate the above integral for $n = 1,4,7$.}\\
For $n = 1$:\\
$\int_0^{\pi/2} \frac{sin(x)}{sin(x) + cos(x)} dx = \frac{\pi}{4}$\\~\\
For $n = 4$:\\
$\int_0^{\pi/2} \frac{sin^4(x)}{sin^4(x) + cos^4(x)} dx = \frac{\pi}{4}$\\~\\
For $n = 7$:\\
$\int_0^{\pi/2} \frac{sin^7(x)}{sin^7(x) + cos^7(x)} dx = \frac{\pi}{4}$\\~\\

\noindent (c) \textit{Evaluate the integral by hand as follows.}\\~\\
(i) Rewrite the integral using the substitution $u = \frac{\pi}{2} - x$.\\
If $u = \frac{\pi}{2} - x$ then $x = \frac{\pi}{2} - u$.\\
The limits then become $\frac{\pi}{2}$ and 0 respectively, but with $du = -dx$ as well, the two negative cancel out and we get:\\
$= \int_0^{\frac{\pi}{2}} \frac{sin^n(\frac{\pi}{2} - u)}{sin^n(\frac{\pi}{2} - u) + cos^n(\frac{\pi}{2} - u)} du$\\
Using the trigonometric identity for angle addition and subtraction:\\
$sin(x-y) = sinxcosy - cosxsiny$ and $cos(x-y) = cosxcosy + sinxsiny$ we get:\\
$\int_0^{\frac{\pi}{2}} \frac{(sin\frac{\pi}{2}cosu - cos\frac{\pi}{2}sinu)^2}{(sin\frac{\pi}{2}cosu - cos\frac{\pi}{2}sinu)^n + (cos\frac{\pi}{2}cosu + sin\frac{\pi}{2}sinu)^n}du$, which simplifies to:\\
$\int_0^{\frac{\pi}{2}} \frac{cos^nu}{cos^nu + sin^nu}du$\\~\\
(ii/iii) Add the integral you obtain above to the original integral.\\
Because the u can be any arbitrary variable, I am going to re-label u as x.\\
Also, I am going to label the original integral as $I$.\\
Therefore:\\
$2I = \int_0^{\frac{\pi}{2}} \frac{cos^n(x)}{cos^n(x)+sin^n(x)} dx + \int_0^{\frac{\pi}{2}} \frac{sin^n(x)}{sin^n(x) + cos^n(x)}dx$\\
$2I = \int_0^{\frac{\pi}{2}} \frac{sin^n(x) + cos^n(x)}{sin^n(x) + cos^n(x)dx}$\\
$2I = \int_0^{\frac{\pi}{2}} 1 dx$\\
$2I = \frac{\pi}{2}$\\
$I = \frac{\pi}{4}$\\
The original integral will always equal $\frac{\pi}{4}$ no matter what n is input.


\end{document}  