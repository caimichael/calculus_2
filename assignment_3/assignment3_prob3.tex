\documentclass[11pt, oneside]{article}   	% use "amsart" instead of "article" for AMSLaTeX format
\usepackage{geometry}                		% See geometry.pdf to learn the layout options. There are lots.
\geometry{letterpaper}                   		% ... or a4paper or a5paper or ... 
%\geometry{landscape}                		% Activate for rotated page geometry
%\usepackage[parfill]{parskip}    		% Activate to begin paragraphs with an empty line rather than an indent
\usepackage{graphicx}				% Use pdf, png, jpg, or eps§ with pdflatex; use eps in DVI mode
								% TeX will automatically convert eps --> pdf in pdflatex		
\usepackage{amssymb}

%SetFonts

%SetFonts


\title{Assignment 3 Problem 3}
\author{Michael Cai}
\date{\today}							% Activate to display a given date or no date

\begin{document}
\maketitle
\noindent (a) $\int \frac{tan^{-1}x}{4x^2}$\\
$= \frac{1}{4} \int x^{-2} tan^{-1}x$\\
Using rule 95 from the Table of Integrals, which states:\\
$\int u^n tan^{-1} u du = \frac{1}{n+1}[u^{n+1}tan^{-1}u - \int \frac{u^{n+1}du}{1+u^2}]$\\
Therefore $\frac{1}{4} \int x^{-2} tan^{-1}x = \frac{1}{4} -1[x^{-1}tan^{-1}x - \int \frac{1}{x(1+x^2)}$\\~\\
Because $\int \frac{1}{x(1+x^2)}$ contains a rational function, I choose to use Partial Fraction Decomposition to evaluate the integral.\\
$\int \frac{1}{x(1+x^2)} = \frac{A}{x} + \frac{Bx+C}{x^2+1}$\\
If we cross-multiply and consider the equality of the numerator of the integrand to the cross multiplied partial fractions we get:\\
$1 = A(x^2+1) + (Bx+C)(x)$\\
Let us consider $x = 0$.\\
If $x = 0$ then $A = 1$.\\
Now if we further expand the RHS to solve for the coefficients we get:\\
$1 = Ax^2 + A + Bx^2 + Cx$\\
Collecting terms we get:\\
$1 = (A+B)x^2 + Cx + A$\\
Therefore, $C = 0$ because the linear term, x, on the LHS has the coefficient of 0.\\
And, $B = -1$ since $A+B$ must equal 0 for the same reason.\\
Thus the completed PFD is $\int \frac{1}{x(1+x^2)} = \int \frac{1}{x} - \frac{x}{1+x^2}$\\
If we separate the integrals and use a simple u-substitution on the second fraction, setting $u = 1+x^2$ and thus $du = 2xdx \rightarrow \frac{1}{2}du = xdx$ then:\\
$\int \frac{1}{x(1+x^2)} = ln|x| + \frac{1}{2}ln|1+x^2| + C$\\
And therefore plugging that back into the original equation to evaluate the entire integral we get:\\
$\frac{1}{4} (ln|x| - \frac{1}{2}ln|1+x^2| - \frac{tan^{-1}x}{x}) + C$\\~\\~\\

\noindent (b) $\int \frac{x}{x^4 + 2x^2 + 5}$\\
First we have to complete the square in the denominator.\\
$x^4 + 2x^2 + 5 = (x^2 + 1) + 4$\\
Thus the integral equals $\int \frac{x}{(x^2+1)^2 + 2^2}$\\
This fits the form of rule 25 in the Table of Integrals if we first use u-substitution to set $u = x^2 + 1$ thus making $du = 2xdx \rightarrow \frac{1}{2}du = xdx$.\\
Therefore the integral equals $\frac{1}{2} \int \frac{du}{u^2 + 2^2} = \frac{1}{2} ln(u + \sqrt{a^2 + u^2}) + C$\\
$= \frac{1}{2} ln(x^2 + 1 + \sqrt{4 + (x^2+1)^2}) + C$\\~\\

\noindent (c) $\int \frac{(x^2-1)^{3/2}}{x}dx$\\
First to make the substitution more apparent, we rewrite the integral as $\int \frac{\sqrt{x^2-1}^3}{x}dx$.\\
Because there is a square root, we know that we must make a trigonometric substitution. The best substitution for the form $\sqrt{u^2-a^2}$ is $x = sec\theta$, which makes $dx = sec\theta tan\theta d\theta$.\\
Therefore the integral simplifies to $\int \frac{tan^3\theta}{sec\theta}sec\theta tan\theta d\theta$, which further simplifies to $\int tan^4\theta d\theta$.\\
Using rule number 75 on the Table of Integrals, we get:\\
$ = \frac{1}{3}tan^3\theta - \int tan^2\theta d\theta$\\
And then again using rule 65 on the Table of Integrals, we get:\\
$ = \frac{1}{3}tan^3\theta - [tan\theta - \theta + C]$\\
Substitution x back into the equation we get:\\
$ = \frac{1}{3}tan^3 sec^{-1} x - [tan sec^{-1} x - sec^{-1} x + C]$\\
We use the formula $tan(sec^{-1}x) = \sqrt{1 - \frac{1}{x^2}}x$ and thus the final form is:\\
$ = \frac{1}{3}(x^2-1)^{3/2} - \sqrt{x^2 - 1} + sec^{-1}x + C$\\

\end{document}  