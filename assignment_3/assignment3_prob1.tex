\documentclass[11pt, oneside]{article}   	% use "amsart" instead of "article" for AMSLaTeX format
\usepackage{geometry}                		% See geometry.pdf to learn the layout options. There are lots.
\geometry{letterpaper}                   		% ... or a4paper or a5paper or ... 
%\geometry{landscape}                		% Activate for rotated page geometry
%\usepackage[parfill]{parskip}    		% Activate to begin paragraphs with an empty line rather than an indent
\usepackage{graphicx}				% Use pdf, png, jpg, or eps§ with pdflatex; use eps in DVI mode
								% TeX will automatically convert eps --> pdf in pdflatex		
\usepackage{amssymb}

%SetFonts

%SetFonts


\title{Assignment 3 Problem 1}
\author{Michael Cai}
\date{\today}							% Activate to display a given date or no date

\begin{document}
\maketitle
%\section{}
%\subsection{}

\noindent 1. Find the area of the region bounded by $y = \frac{3x^2+x}{(x^2+1)(x+1)}$, $y=0$, $x=0$, and $x=1$.\\~\\
For the integral $\int_0^1 \frac{3x^2+x}{(x^2+1)(x+1)}$ we first use partial fraction decomposition because the integrand is a rational function (a ratio of polynomials).\\
Thus $\int_0^1 \frac{3x^2+x}{(x^2+1)(x+1)} = \frac{Ax + B}{x^2+1} + \frac{C}{x+1}$\\
We then cross multiply and consider the equality of the numerator of the integrand with the cross-multiplied terms of the partial fractions.\\
$3x^2 + x = (Ax + B)(x+1) + C(x^2+1)$\\
If $x = -1$ then $C=1$.\\
We then expand the RHS to solve for the match the coefficients on the LHS and the RHS.\\
$3x^2 + x = Ax^2 + Ax + Bx + B + Cx^2 + C$\\
$3x^2 + x = (A+C)x^2 + (A+B)x + B+C$\\
Since the constant term on the LHS is 0, then $B+C = 0 \rightarrow B = -1$\\
Since the coefficient of the linear term, $x$, on the LHS = 1 then $1 = A+B \rightarrow A = 2$\\
Therefore: $\frac{3x^2+x}{(x^2+1)(x+!)} = \frac{2x-1}{x^2+1} + \frac{1}{x+1}$\\
$\int_0^1 \frac{3x^2+x}{(x^2+1)(x+1)} = \int_0^1 \frac{2x-1}{x^2+1} + \int_0^1 \frac{1}{x+1}$\\
$= \int_0^1 \frac{2x}{x^2+1} - \int_0^1 \frac{1}{x^2+1} + \int_0^1 \frac{1}{x+1}$\\~\\
Consider the $\int_0^1 \frac{2x}{x^2+1}$ term:\\
We use u-substitution and set $u= x^2 + 1$ and thus $du = 2xdx$ and the limits become $u = 2$ and $u = 1$.\\
Consider the $\int_0^1 \frac{1}{x+1}$ term:\\
Here we also use u-substitution and set $u = x+1$ and thus $du = dx$ with the limits becoming $u = 2$ and $u = 1$.\\
Now if we integrate all 3 of the separate terms, we get get two with natural logs and one with arctan.\\
$= \left. ln|u| \right|_1^2 - \left. tan^{-1}(x) \right|_0^1 + \left. ln|u| \right|_1^2$\\
$= 2ln|2| - \frac{\pi}{4}$


\end{document}  