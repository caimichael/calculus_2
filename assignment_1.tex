\documentclass[12pt]{article}

\usepackage[margin = 0.75in, paperwidth = 8.5 in, paperheight=11in]{geometry}
\usepackage{pgfplots}

\begin{document}

\title{Assignment 1}
\date{\today}
\author{Michael Cai}

\maketitle{}

\section{Exercise One}

\textit{Let $R(x) = \int_{0}^{x}\sqrt{1+t^2}dt$. Answer the following questions with justification.}\\

(a) \textit{Evaluate $R(0)$ and determine if $R(x)$ is an even or an odd function.}\\
\\An integral is a tool to find the area bounded by the x-axis and the function represented by the integrand, which is bounded by the lower and upper limits of the integral. Therefore the integral of any function with the lower limit set equal to the upper limit would be equal to 0. This is because the area bounded would actually be a one-dimensional line. \\

Therefore \textbf{$R(0) = \int_{0}^{0} \sqrt(1+t^2) dt$ is equal to zero.} \\
\\
To determine if $R(x)$ is an even or an odd function, I refer to the theorem of \underline{Integrals of Symmetric Functions:}
\\
Suppose $f$ is continuous on $[-a,a]$: \\
(a) If $f$ is even $[f(-x) = f(x)]$, then $\int_{-a}^{a}f(x)dx = 2\int_{0}^{a}f(x)dx$ \\
(b) If $f$ is odd $[f(-x) = -f(x)]$, then $\int_{-a}^{a}f(x)dx = 0$ \\
\\
To determine whether $R(x)$ is an odd or an even number, we must compare the outcomes of R given opposite inputs, x and -x. \\
For example, compare $R(1) = \int_{0}^{1} \sqrt{1+t^2}dt$ to $R(-1) = \int_{0}^{-1} \sqrt{1+t^2}dt$ which equals $-\int_{-1}^{0} \sqrt{1+t^2}dt$. \\
\\
(d) \textit{The graph clearly demonstrates the function:}\\
%\begin{tikzpicture}
%  \begin{axis}[ 
%    xlabel=$t$,
%    ylabel=$f(x) = \sqrt{1+t^2}$
%  ] 
%    \addplot {\sqrt{x^2}}; 
%  \end{axis}
%\end{tikzpicture}
\\
The integrand is a symmetric function; however since $R(-1)$ causes the upper limit of the integral to be lower than the lower limit, you must evaluate $R(-1) = \int_{0}^{-1} \sqrt{1+t^2} dt $ as if it were $-\int_{0}^{1}\sqrt{1+t^2}dt$. Therefore $R(-1) = -R(1)$, and thus \textbf{$R(x)$ is an odd function.}
\\

\noindent (b) \textit{Is $R(x)$ increasing or decreasing?}\\
Because we have established that $R(x)$ is an odd function, we know that the value of $R(-x) = -R(x)$.\\
When $x>0$, $R(x)$ is increasing and positive because the integrand is increasing and positive.\\
Because $R(x)$ is an odd function, when $x<0$, $R(x)$ is decreasing and negative because the integrand is increasing and positive. \\
Therefore \textbf{$R(x)$ is increasing} because as $x$ increases, $R(x)$ becomes less negative, and as it crosses 0, $R(x)$ becomes increasingly positive. \\

\noindent (c) \textit{What can you say about the concavity of $R(x)$?}\\
To find the concavity of $R(x)$ we must observe if $R''(x)$ is positive or negative. \\
By the Fundamental Theorem of Calculus Part 1, we know that $R(x) = \int_{0}^{x}f(t)dt \Rightarrow R'(x) = f(x)$, which further implies that $R''(x) = f'(x)$.\\
\\
$f'(x) = \frac{d}{dx}(\sqrt{1+x^2}) = \frac{1}{2}(1+x^2)^{-\frac{1}{2}}*2x$\\
= $x(1+x^2)^{-\frac{1}{2}}$\\
=$\frac{x}{\sqrt{1+x^2}}$ \\

%\begin{tikzpicture}
%  \begin{axis}[ 
%    xlabel=$x$,
%    ylabel=$f(x) = \frac{x}{\sqrt{1+x^2}}$
%  ] 
%    \addplot function {\frac{x}{\sqrt{1+x^2}}; 
%  \end{axis}
%\end{tikzpicture}

\noindent Because $f'(x)$ is strictly increasing, then \textbf{$R(x)$ is concave up.}\\
\\
(e) \textit{Find $\lim_{x\to\infty}\frac{R(x)}{x^2}$} \\



\end{document}

